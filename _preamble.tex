\usepackage[english]{babel}
\usepackage[T1]{fontenc}
\usepackage{graphicx}
\usepackage{hyperref} % \hypersetup{hidelinks=true}  % avoid colored box in links
\usepackage{multirow}
\usepackage{tabularx}
\usepackage{caption}

% redefine the X column to be based on a m-column instead of a p-column
\renewcommand\tabularxcolumn[1]{m{#1}}

\usepackage{longtable}
\usepackage[table]{xcolor}
\usepackage{fancyhdr}
\usepackage{lipsum}
\usepackage{tcolorbox}

% Select Arial font for the whole document
\usepackage{helvet}
\renewcommand{\familydefault}{\sfdefault}

% Document geometry
\usepackage[a4paper, includehead,
  textwidth=16cm, textheight=23cm,
  headheight=2.7cm,
  includemp,
  marginparsep=0.cm,
  marginparwidth=0.cm,
  %showframe,
]{geometry}


\usepackage{enumerate}
\usepackage[shortlabels]{enumitem}
\usepackage{tocloft}

% AD and RD entries
\newenvironment{ADlist}
               {\begin{enumerate}[start=1,label={AD\arabic*}]}  % \bfseries
               {\end{enumerate}}               
\newenvironment{RDlist}
               {\begin{enumerate}[start=1,label={RD\arabic*}]}  % \bfseries
               {\end{enumerate}}
\newcommand{\ARDitem}[2]{\item \label{#1} {#2}}
\newcommand{\citedoc}[1]{[\ref{#1}]}


% List of requirements
\newcommand{\listreqname}{List of Requirements:}
\newlistof{req}{toreq}{\listreqname}
\newcommand{\reqlong}[3][req:\thereq]{%
\refstepcounter{req}
\par\vspace{0.2cm}\noindent\textbf{\ESOreqprefix{}\thereq:} #3 \label{#1}\vspace{0.2cm}
\addcontentsline{toreq}{req}
{\protect \textbf{\ESOreqprefix{}\numberline{\thereq}}#2}\par}
\newcommand{\req}[2][req:\thereq]{\reqlong[#1]{#2}{#2}}
\newcommand{\citereq}[1]{\textbf{\ESOreqprefix{}\ref{#1}}}


% List of questions
\newcommand{\listquestionname}{List of Questions:}
\newlistof{question}{toquestion}{\listquestionname}
\newcommand{\questionlong}[3][q:\thequestion]{%
\refstepcounter{question}
\par\vspace{0.2cm}\noindent\textbf{Q\thequestion:} #3 \label{#1}\vspace{0.2cm}
\addcontentsline{toquestion}{question}
{\protect \textbf{Q\numberline{\thequestion}}#2}\par}
\newcommand{\question}[2][q:\thequestion]{\questionlong[#1]{#2}{#2}}
\newcommand{\citequestion}[1]{\textbf{Q\ref{#1}}}

